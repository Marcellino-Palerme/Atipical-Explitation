\documentclass[english]{article}
\usepackage[T1]{fontenc}
\usepackage[utf8]{inputenc}
\usepackage{lmodern}
\usepackage[a4paper]{geometry}
\usepackage{babel}
\usepackage{fp}
\usepackage{pgf-pie}

% Define number image by symptom
\def \nbAlt {988}
\def \nbBig {665}
\def \nbMac {750}
\def \nbMil {100}
\def \nbMyc {768}
\def \nbPse {780}
\def \nbSym {857}

% Calculate number total of recto images
\FPadd\ttImgR\nbAlt\nbBig
\FPadd\ttImgR\ttImgR\nbMac
\FPadd\ttImgR\ttImgR\nbMil
\FPadd\ttImgR\ttImgR\nbMyc
\FPadd\ttImgR\ttImgR\nbPse
\FPadd\ttImgR\ttImgR\nbSym
\FPeval\ttImgR{round(ttImgR:0)}

% Calculate number total of images
\FPadd\ttImgRV\ttImgR\ttImgR
\FPeval\ttImgRV{round(ttImgRV:0)}

% Calculate proportion of each symptom
\FPdiv\pAlt\nbAlt\ttImgR
\FPmul\pAlt\pAlt{100}
\FPeval\pAlt{round(\pAlt:2)}
\FPdiv\pBig\nbBig\ttImgR
\FPmul\pBig\pBig{100}
\FPeval\pBig{round(\pBig:2)}
\FPdiv\pMac\nbMac\ttImgR
\FPmul\pMac\pMac{100}
\FPeval\pMac{round(\pMac:2)}
\FPdiv\pMil\nbMil\ttImgR
\FPmul\pMil\pMil{100}
\FPeval\pMil{round(\pMil:2)}
\FPdiv\pMyc\nbMyc\ttImgR
\FPmul\pMyc\pMyc{100}
\FPeval\pMyc{round(\pMyc:2)}
\FPdiv\pPse\nbPse\ttImgR
\FPmul\pPse\pPse{100}
\FPeval\pPse{round(\pPse:2)}
\FPdiv\pSym\nbSym\ttImgR
\FPmul\pSym\pSym{100}
\FPeval\pSym{round(\pSym:2)}

\title{Identification automatisée de multiples symptômes foliaires par vision numérique : apport des approches multi-vues}
\begin{document}

\maketitle

\tableofcontents

\section{Abstract}

\section{Materials and Method}

\subsection{General}
\ttImgR
\\
\ttImgRV\\
\pAlt\\
\pBig\\
\pMac\\
\pMil\\
\pMyc\\
\pPse\\
\pSym
\subsection{Image database}
The leaf’s shooting is standardized. We made a portable stand. (add photo) It composed of  a base  onto we fixed at left and right two LED light supports and at up a camera support. The LED lights are oriented to the middle of base.
\\
The image composition is standardized too. The background is uniform blue. At left, there is a ruler. At right, there are three color mire (black, gray and white). At down, we find photo’s name and orientation leaf (recto or verso). And at the middle, we put the leaf.

\begin{tikzpicture}
	
	\pie[
	text = legend
	]{\pAlt/Alt,
	  \pBig/Big,
	  \pMac/Mac,
	  \pMil/Mil,
	  \pMyc/Mic,
	  \pPse/Pse,
      \pSym/Sym}
	
\end{tikzpicture}
\end{document}
